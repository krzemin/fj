\documentclass{article}[12pt]
\usepackage[sc]{mathpazo}
\linespread{1.05}
\usepackage[T1]{fontenc}
\usepackage[utf8]{inputenc}
\usepackage{anysize}
\marginsize{1.8cm}{1.8cm}{1.8cm}{1.8cm}
\usepackage[hidelinks]{hyperref}
\usepackage{minted}
\usepackage{tcolorbox}
\usepackage{etoolbox}
\BeforeBeginEnvironment{minted}
  {\begin{tcolorbox}[colback=white,arc=0mm,enlarge top by=1mm,enlarge bottom by=1mm,enlarge left by=8mm,width=\linewidth-16mm]}
\AfterEndEnvironment{minted}
  {\end{tcolorbox}}
\usemintedstyle{friendly}
\usepackage{upquote}
\expandafter\def\csname PYGdefault@tok@err\endcsname{
  \def\PYGdefault@bc##1{{\strut ##1}}
}

\author{Piotr Krzemiński}
\title{Implementing type erasure based on Featherweight Java}
\date{}
\begin{document}
\maketitle


\begin{abstract}
Featherweight Java is a minimal core calculus describing the Java
language type system. It provides two calculi named \emph{FJ} --
for plain classes with fields and methods, and \emph{FGJ} -- extending
the first one with generic types. Type erasure is expressed as
a translation from \emph{FGJ} to \emph{FJ} that preserves appropriate
properties around type-checking and evaluation semantics. In this
article we review implementation\footnote{The implementation is available
at \url{https://github.com/krzemin/type_erasure_featherweight_java}}
of these calculi realized in \emph{Scala} and we provide comprehensive
examples demonstrating type erasure in action.
\end{abstract}


\section{Introduction}

Certain class-based programming languages provide a concept of generic
types. It enables parameterizing classes with type parameters.
Such a feature makes it possible to write polymorphic code that can
work with arbitrary actual type arguments. One of the key applications
of generic types can be found in standard libraries, e.g.,
collections and algorithms working with them. For instance, a
sorting algorithm can be implemented as a function taking
collection of elements and returning sorted collection, with no
particular concern about type of elements inside the collection, as
soon as it knows how to compare them.

There are several possible implementations of generics, including:

\begin{itemize}

\item{\textbf{type passing}} -- it preserves information about
  type parameters at runtime, which allows to distinguish
  for example \textbf{List<Integer>} from \textbf{List<String>}; 
  this implementation is chosen in \emph{.NET} languages like
  \emph{C\#}

\item{\textbf{type instantiating}} -- for every instantiation of
  parameterized class with actual type arguments, a separate
  class is generated that maintains no information about generic
  types -- for example \textbf{List\$Integer} and \textbf{List\$String};
  we can still distinguish between them, but no type parameter
  information is present at runtime. This implementation is present
  in \emph{C++} templates

\item{\textbf{type erasure}} -- it eliminates information about
  type parameters at compilation time, replacing them with their
  so-called \emph{type bounds}; at runtime we have only single
  \textbf{List<Object>} class which can hold any elements; we cannot
  distinguish lists of integers from lists of strings in this
  implementation. Type erasure is used by the \emph{Java} language.

\end{itemize}


In this article we will review implementation of two small
programming languages that imitate subsets of Java, being
syntactically compatible with the full language, defined in
\cite{fj}. These two languages are:

\begin{itemize}
\item{\textbf{FJ}} -- minimal subset of Java with classes,
fields and methods only
\item{\textbf{FGJ}} -- the language extended with
type-parameterized classes and methods.
\end{itemize}


We will define syntax, look at the examples and express type
erasure as translation from \emph{FGJ} to \emph{FJ} that
preserves some important properties about types and behaviour.
We will not explain all the details of provided erasure
implementation, but instead will look at the example programs
and their erased version to see type erasure in action. Reading
the \emph{Featherweight Java} paper is not absolutely required,
although highly recommended for readers willing to deeper
understand erasure rules, where they are clearly defined and
comprehensively explained.

The implementation is written in Scala 2.11.
\textbf{Java JDK}\footnote{
\url{http://www.oracle.com/technetwork/java/javase/downloads/index.html}
} and \textbf{SBT} \footnote{
\url{http://www.scala-sbt.org}
} are required to be installed in order to run the examples.

\section{Featherweight Java}

We start from introduction of the \emph{Featherweight Java} calculus at
the abstract level and see how to encode simple programs in \emph{FJ}
and \emph{FGJ}.

\subsection{Idea}

Looking for a tool to precise describing the Java type system, we need
to focus on modelling only those parts of the language which are
important from the type system perspective while omitting those
which are not. Trying to model full Java in a formal way would result
in an enormous calculus being hard to grasp. Therefore
\emph{Featherweight Java} favours compactness over completeness,
providing only few combinators, while still being a legal subset
of Java, only little larger than the original $\lambda$-calculus.

Compactness of the \emph{FJ} is achieved by reducing the language
heavily. It comes down to:

\begin{itemize}
\item no concurrency primitives like \texttt{synchronized} keyword
\item no reflection
\item no interfaces
\item no method overloading
\item no inner classes
\item no static members
\item no member access control -- all methods and fields are public
\item no primitive types
\item no null pointers
\item no assignments/setters
\end{itemize}
Instead, we focus only on minimal language subset, including:

\begin{itemize}
\item mutually recursive class definitions
\item object creation
\item field access
\item method invocation, overriding and recursion through \texttt{this}
\item subtyping
\item casting
\end{itemize}


\subsection{Syntax}

Let's start with a simple example -- an immutable \texttt{Pair} class
definition. We define two plain classes -- \texttt{A} and \texttt{B},
and class \texttt{Pair} containing members \texttt{fst} and \texttt{snd}
and method \texttt{setfst} that returns new instance of a
pair with modified first element.

\begin{minted}[mathescape]{java}
class A extends Object {
  A() { super(); }
}
class B extends Object {
  B() { super(); }
}
class Pair extends Object {
  Object fst;
  Object snd;
  Pair(Object fst, Object snd) {
    super(); this.fst = fst; this.snd = snd;
  }
  Pair setfst(Object newfst) {
    return new Pair(newfst, this.snd);
  }
}
\end{minted}

\emph{FJ} is a class-based language where we can define classes like in
Java, but satisfying some constraints:

\begin{itemize}
\item we always write the super class name, even if it's trivial
  (\texttt{Object})
\item we always write the receiver of field or method, even if it's
  trivial (\texttt{this})
\item \texttt{this} is simply a variable rather than a keyword, unlike
  in full Java
\item we always write constructor which initializes all fields
  defined in that class and call \texttt{super} which refers to the
  super class constructor, which initializes its fields, etc.
\item constructors are the only place where \texttt{super} or
  \texttt{=} appears
\end{itemize}

\subsubsection{Expressions}

In FJ we have 5 types of expressions, which can appear in methods body:

\begin{itemize}
\item{\textbf{variable access}} -- \texttt{newfst} or reference to
  \texttt{this}
\item{\textbf{object construction}} -- \texttt{new A()},
  \texttt{new B()} or \texttt{new Pair(newfst, this.snd)}
\item{\textbf{field access}} -- in \texttt{this.snd} expression 
  a field named \texttt{snd} is accessed on the object referred by
  a variable \texttt{this}
\item{\textbf{method invocation}} -- \texttt{e3.setfst(e4)}
  is an example of invocation of method \texttt{setfst} on object
  \texttt{e3} with single argument \texttt{e4}
\item{\textbf{casts}} -- \texttt{(A)(new Pair(new A(), new B()).fst)}
  is an example of type cast used to recover type information about
  \texttt{fst} field
\end{itemize}

\subsubsection{Programs}

\emph{FJ} programs consist of a class table and an expression
to be evaluated, strictly corresponding to static \texttt{main} method in
executable Java classes. We intuitively expect that in the context
of previously defined classes \texttt{A}, \texttt{B} and \texttt{Pair},
an expression

\begin{minted}[mathescape]{java}
new Pair(new A(), new B()).setfst(new B())
\end{minted}
will eventually evaluate to
\begin{minted}[mathescape]{java}
new Pair(new B(), new B())
\end{minted}


\subsection{Extending with generic types}

Let's extend the \emph{FJ} calculus with generic types. We parameterize
class \texttt{Pair} with two type parameters \texttt{X} and \texttt{Y}.
Using them, we encode types of fields \texttt{fst} and \texttt{snd},
accordingly. Notice that in definition of method \texttt{setfst} we
introduce another type parameter \texttt{Z}, which allows to return
an instance of a pair with not only value of type \texttt{X} modified,
but allowing also to change first element's type!

\begin{minted}[mathescape]{java}
class Pair<X extends Object, Y extends Object> extends Object {
  X fst;
  Y snd;
  Pair(X fst, Y snd) {
    super(); this.fst = fst; this.snd = snd;
  }
  <Z extends Object> Pair<Z, Y> setfst(Z newfst) {
    return new Pair<Z, Y>(newfst, this.snd);
  }
}
\end{minted}
More generally, the syntax is extended with:

\begin{itemize}
\item type parameters lists for classes and methods -- in the example
  above \texttt{X} and \texttt{Y} are type parameters for class
  \texttt{Pair}, while \texttt{Z} is a type parameter of method
  \texttt{setfst}
\item every type parameter has to be bounded by some actual class
  type, possibly parameterized with type variables, e.g.,
  \texttt{X extends C<X>}
\item in contrast to Java we always write the bound even if it
  is \texttt{Object}
\item object construction and method invocation both take
  type arguments list like \texttt{new Pair<Z, Y>(...)} or
  \texttt{.setfst<B>(...)}, but empty parameter lists (\texttt{<>})
  can be omitted
\end{itemize}
Our refined example program looks as follows.

\begin{minted}[mathescape]{java}
new Pair<A,B>(new A(), new B()).setfst<B>(new B())
\end{minted}
And it evaluates to expression
\begin{minted}[mathescape]{java}
new Pair<B,B>(new B(), new B())
\end{minted}

\subsection{Type erasure as translation from FGJ to FJ}

We can express type erasure as compilation from \emph{FGJ} syntax
to \emph{FJ} by replacing all type variables with their bounds
and inserting some number of casts, when needed to smartly recover
type information from the original \emph{FGJ} code. The example class
\texttt{Pair<X, Y>} after erasure looks exactly like the previous
\texttt{Pair} class without generic types. Similarly, the following
expression:
\begin{minted}[mathescape]{java}
new Pair<A,B>(new A(), new B()).snd
\end{minted}
erases to:
\begin{minted}[mathescape]{java}
(B)new Pair(new A(), new B()).snd
\end{minted}
Notice that the cast to \texttt{B} was inserted to restore type
of \texttt{snd} field which is annotated with type \texttt{Object}
in the erased \texttt{Pair}.

\section{Implementation review}

Having gradual introduction to \emph{FJ} and \emph{FGJ} behind,
let's get sight of the Scala implementation of these small languages.

\subsection{FJ module}

\emph{FJ}-related code is contained in \texttt{src/main/scala/fj}
directory. There are syntax for \emph{FJ} programs defined in
\texttt{AST.scala}, type checker in \texttt{Types.scala} and
evaluator in \texttt{Eval.scala}.

\subsubsection{Syntax}

Classes, fields and methods are represented by the following set of
case classes.

\begin{minted}[mathescape]{scala}
type VarName = String
type TypeName = String

case class Class(name: TypeName,
                 baseClass: TypeName,
                 fields: List[Field],
                 methods: List[Method])

case class Field(name: VarName,
                 fieldType: TypeName)

case class Method(name: VarName,
                  resultType: TypeName,
                  args: List[Argument],
                  body: Expr)

case class Argument(name: VarName,
                    argType: TypeName)
\end{minted}

We represent \emph{FJ} classes using 4 nested data structures
which hold all necessary information about base classes, fields
and methods. There are type aliases defined for type and variable
names, internally represented as strings. Similarly, we encode
expressions using \texttt{Expr} trait and the following case classes
extending it:

\begin{minted}[mathescape]{scala}
trait Expr

case class Var(name: VarName) extends Expr

case class FieldAccess(expr: Expr,
                       fieldName: VarName) extends Expr

case class Invoke(expr: Expr,
                  methodName: VarName,
                  args: List[Expr]) extends Expr

case class New(className: TypeName,
               args: List[Expr]) extends Expr

case class Cast(className: TypeName,
                expr: Expr) extends Expr
\end{minted}

In actual implementation all those classes have overridden
method \texttt{toString} which prettifies syntax of our
programs when printing to the console.

\subsubsection{Example encoded using Scala syntax}

Let's review how we can encode our first example with
\texttt{Pair} class implementation.

\begin{minted}[mathescape]{scala}
val A = Class("A", "Object")
val B = Class("B", "Object")
val Pair = Class(
  name = "Pair",
  baseClassName = "Object",
  fields = List(
    Field("fst", "Object"),
    Field("snd", "Object")
  ),
  methods = List(
    Method(
      name = "setfst",
      resultType = "Pair",
      args = List(Argument("newfst", "Object")),
      body = New("Pair", List(
        Var("newfst"), FieldAccess(Var("this"), "snd")
      ))
    )
  )
)
\end{minted}
It's just straightforward rewriting our \texttt{Pair} class
with two fields and one method. We represent class tables and
programs as follows.

\begin{minted}[mathescape]{scala}
type ClassTable = Map[TypeName, Class]
case class Program(classTable: ClassTable, main: Expr)

val classTable: ClassTable = buildClassTable(List(A, B, Pair))
val main: Expr = Invoke(
  New("Pair", List(New("A"), New("B"))),
  "setfst",
  List(New("B"))
)
val program = Program(classTable, main)
\end{minted}

We have helper function \texttt{buildClassTable} which takes a list
of classes and returns a class table built out of them.
\texttt{Program} is just, following definition, paired class table
with main expression to be evaluated.

\subsubsection{Type checker}

There are type-checking rules provided in the \emph{FJ} paper,
which are implemented in \texttt{fj.Types}.

Subtyping in \emph{FJ} is the reflexive and transitive closure of
inheritance relation between classes. It can be decided only by
looking at the class table. Implementation of subtyping is provided
as a recursive function \texttt{fj.Types.isSubtype}.

Main type-checking function is \texttt{fj.Types.exprType} that
finds concrete type of expression in given typing context $\Gamma$
or indicates that the expression is incorrectly-typed. Context
$\Gamma$ contains information about actual types of the available
variables and is represented as \texttt{Map[VarName, SimpleType]}.
There is also an auxiliary function \texttt{fj.Types.progType}
that type-checks a whole program, ensuring that all classes, fields,
methods are well-typed according to the typing rules, and returns
type of the main expression.

\subsubsection{CBV Evaluator}

In \cite{fj} there were reduction rules given for expressions in
the form of so-called \emph{operational semantics}, which doesn't
precise the order of evaluation. Trying to implement the expression
evaluator, some evaluation strategy has to be chosen. This
implementation is realized with \emph{call by value} semantics,
which corresponds to that known from the full Java, where
the method's arguments are evaluated from left to right.

From the \emph{FJ} calculus point of view when some reduction error
occurs (like trying to create object of unknown class or trying to
invoke non-existing method), such a configuration is called
\emph{stuck} and the evaluation process cannot be continued.
In this implementation we don't bother too much about error
handling in the interpreter. When some errored configuration is
detected, simply \texttt{RuntimeException} is thrown with
an appropriate error message, forgetting about the result
computed so far.

The evaluator is rather simple adaptation of the reduction rules to
\emph{call by value} strategy. It can be found at
\texttt{fj.Eval.evalExpr} for evaluation expressions in the given
context (i.e. the class table) and auxiliary function
\texttt{fj.Eval.evalProg} which takes the program, builds the
class table and evaluates its main expression.

\subsubsection{Running examples}

Let's consider context of two previous code snippets. Below
we present how the type-checker and program evaluator could be
launched.

\begin{minted}[mathescape]{scala}
println(program.main) // prints: new Pair(new A(), new B()).setfst(new B())
fj.Types.programType(program) // Some(Pair) - type of main expression
val result = fj.Eval.evalProg(program) // New("Pair", List(New("B"), New("B")))
println(result) // prints: new Pair(new B(), new B())
\end{minted}
Similar example can be run from console by typing

\begin{verbatim}
sbt "runMain fj.examples.Pairs"
\end{verbatim}

Let's encode some more interesting program in our language. In
the \emph{FJ} we don't have primitive types, especially numbers.
But there is a way to encode the natural numbers using just classes
and objects, similarly to \emph{Church numerals} in
the $\lambda$-calculus, but instead of folding functions, we
will fold objects of class \texttt{Succ} $n$ times over the
instance of class \texttt{Zero} to represent number $n$.

\begin{minted}[mathescape]{scala}
class Nat extends Object {
  Nat() { super(); }
  Nat succ() { return new Succ(this); }
  Nat plus(Nat n) { return n; }
}

class Zero extends Nat { 
  Zero() { super(); }
}

class Succ extends Nat {
  Nat prev;
  Succ(Nat prev) { super(); this.prev = prev; }
  Nat plus(Nat n) { return this.prev.plus(n.succ()); }
}
\end{minted}

Following the definition, we represent 0 as \texttt{new Zero()},
1 as \texttt{new Succ(new Zero())},
2 as \texttt{new Succ(new Succ(new Zero()))}, and so on.
Addition is implemented as a recursive function with base case
at 0 (indeed, $0 + n = n$). Recursive step is in the class
\texttt{Succ} -- it transforms general addition $m + n$ into
$(m-1) + (n+1)$ until base case for $m = 0$ is reached.
Method \texttt{succ} is implemented in the base class
\texttt{Nat} as wrapping the actual number \texttt{this}
into object of \texttt{Succ} class.

There is one subtlety connected with implementation of method
\texttt{plus}. Base case of recursion has to be implemented in
the class \texttt{Nat} to satisfy type-checking of the
\texttt{Succ} class. In the full Java we would probably
defined this method as an abstract in the \texttt{Nat} class
and provide two actual implementations in \texttt{Zero} and 
\texttt{Succ}. But in the \emph{FJ} we don't have abstract
methods and without method \texttt{plus} declared in
the \texttt{Nat} class, type checking of recursive invocation 
\texttt{this.prev.plus(...)} in the \texttt{Succ} class
would fail.

This example can be found at \texttt{fj.examples.Numbers}
and launched by typing:

\begin{verbatim}
sbt "runMain fj.examples.Numbers"
\end{verbatim}


\subsection{FGJ module}

\emph{FGJ}-related code is contained in \texttt{src/main/scala/fgj}
directory. There are syntax for \emph{FGJ} programs defined in
\texttt{AST.scala} and extended type checker in \texttt{Types.scala}. 
Unlike to the \emph{FJ} module, there is no program
evaluator\footnote{Such a direct evaluator would have to use type
passing implementation of generics. Actually, extending the \emph{FJ}
evaluator to maintain additional environment for the type variables
and their actual denotations is rather straightforward and might
be considered as an exercise for the reader. The operational
semantics for \emph{FGJ} is defined in \cite{fj}.} available
in this implementation.

\subsubsection{Types}

As the types are now part of our classes, methods and expressions
AST, let's review them first.

\begin{minted}[mathescape]{scala}
type TypeVarName = String

trait Type

case class TypeVar(name: TypeVarName) extends Type

case class ClassType(className: TypeName,
                     argTypes: List[Type]) extends Type
\end{minted}

We have new type alias \texttt{TypeVarName} for the type variables
(again, internally just strings). The types have two possible forms:
\emph{type variables} and \emph{class types} parameterized by
some number of types (which again can be type variables or
class types).

\subsubsection{Classes}

\begin{minted}[mathescape]{scala}
case class BoundedParam(typeVar: TypeVar,
                        boundClass: ClassType)

case class Class(name: TypeName,
                 typeParams: List[BoundedParam],
                 baseClass: ClassType,
                 fields: List[Field],
                 methods: List[Method])

case class Field(name: VarName, fieldType: Type)

case class Method(name: VarName,
                  typeParams: List[BoundedParam],
                  resultType: Type,
                  args: List[Argument],
                  body: Expr)
                  
case class Argument(name: VarName, argType: Type)
\end{minted}

\texttt{BoundedParam} corresponds to single
\texttt{Z extends Object} from the example \texttt{Pair} class.
It is definition
of type variable, bounded by some class type. Notice that we
can write recursive type expression in bounds
(like \texttt{X extends C<X>}) thanks to definition of the class
types, which can be parameterized with arbitrary types.
Classes are parameterized by a list of \emph{bounded parameters}.
Notice change in the \texttt{baseClass} signature which is not
only a~name reference any more, but became the class type
parameterizable by the type variables, like in the example below --
where super-class type \texttt{Collection} is parameterized with
a type variable \texttt{X}.

\begin{minted}[mathescape]{scala}
class List<X extends Object> extends Collection<X> { ... }
\end{minted}

Methods also can be parameterized with type variables as well.
We can use them to encode method's return type and argument
types.

\texttt{ClassTable} and \texttt{Program} definitions are
straightforwardly adjusted to use the refined types.

\subsubsection{Expressions}

AST for the expressions is mostly untouched. Just like in
the \emph{FJ}, we have 5 different forms of them.

\begin{minted}[mathescape]{scala}
trait Expr

case class Var(name: VarName) extends Expr

case class FieldAccess(expr: Expr,
                       fieldName: VarName) extends Expr

case class Invoke(expr: Expr,
                  methodName: VarName,
                  typeArgs: List[Type],
                  args: List[Expr]) extends Expr

case class New(classType: ClassType,
               args: List[Expr]) extends Expr

case class Cast(classType: ClassType,
                expr: Expr) extends Expr
\end{minted}

Occurrences of the \texttt{TypeName} in the classes \texttt{New}
and \texttt{Cast} were replaced by a \texttt{ClassType}. It is possible
to create an object \texttt{new Pair<X,Y>(...)}, but it's illegal
to construct such an expression \texttt{new X()}, in the context
where \texttt{X} and \texttt{Y} are the type variables. Similar
restriction concerns the casting target type too. In the class \texttt{Invoke} there is additional parameter \texttt{typeArgs}
which encodes actual type arguments of the method invocation.

\subsubsection{Type checker}

In the \emph{FGJ} type-checking rules are a bit more complicated.
First of all, subtyping is not a relation between class names any
more, but is generalized for all type forms, including the type
variables. Therefore we differentiate two separate relations:

\begin{itemize}
\item{\textbf{subclassing}} -- it corresponds to the \emph{FJ}'s
  subtyping, can be decided only using class table
\item{\textbf{subtyping}} -- generalized relation between all
  the types that can be decided using additional environment $\Delta$
  that maps type variables to their bounds, where bounds are
  just class types with actual type arguments given.
\end{itemize}

\paragraph{Covariant method overriding}

Unlike in the \emph{FJ}, where we allowed method overriding only with
corresponding (i.e. identical) signatures, covariant method overriding
on the method's result type is allowed in the \emph{FGJ}. The result
type of a method may be a subtype of the result type of the
corresponding method in the superclass, although the bounds of
the type variables and the argument types must be identical (modulo
renaming the type variables).

Function for typing expressions is located at
\texttt{fgj.Types.exprType}, which takes the expression, the class
table and two contexts $\Gamma$ and $\Delta$. Again, we have
auxilliary \texttt{fgj.Types.programType} which checks also
well-typedness of the classes and methods.


\subsection{Type erasure}

Implementation of the type erasure rules is located at
\texttt{src/main/scala/erasure} directory.

\subsubsection{Overview}

The general idea of the type erasure is to translate \emph{FGJ}
programs into \emph{FJ} ones. To perform that task, we have to
implement erasure for all the parts of our programs. Wanting to adopt
the erasure rules from \cite{fj}, several functions are defined,
namely to translate:

\begin{itemize}
\item FGJ types to FJ types -- \texttt{erasure.Erasure.eraseType}
\item FGJ expressions to FJ expressions --
  \texttt{erasure.Erasure.eraseExpr}
\item FGJ classes to FJ classes -- \texttt{erasure.Erasure.eraseClass}
\end{itemize}
And finally, auxiliary function that merges results and translates a
whole \emph{FGJ} program to an \emph{FJ} program is
\texttt{erasure.Erasure.eraseProgram}.

\subsubsection{Examples}

Instead of exploring the implementation details, let's catch some more
interesting \emph{FGJ} programs and their erased versions to see
the rules in action.

\paragraph{Example 1} natural numbers revisited

This is an extended version of natural numbers implementation in
\emph{FGJ}.

\begin{minted}[mathescape]{java}
class Summable<X extends Object> extends Object {
  X plus(X other) { return other; }
}

class Nat extends Summable<Nat> {
  Nat() { super(); }
  Succ succ() { return new Succ(this); }
}

class Zero extends Nat { 
  Zero() { super(); }
}

class Succ extends Nat {
  Nat prev;
  Succ(Nat prev) { super(); this.prev = prev; }
  Nat plus(Nat n) {
    return this.prev.plus(n.succ());
  }
}
\end{minted}

We introduced a class \texttt{Summable<X>} which have one method
\texttt{plus}. In Java we would probably make this class an
interface, but in the \emph{FGJ} we don't have interfaces, so we have
to provide the default implementation returning some value of type
\texttt{X}. Fortunately, we have a parameter of type \texttt{X}
available, so we use it as a return value. It turns out that it
is still a valid implementation of \texttt{plus} for class
\texttt{Zero}, so we don't have to re-implement it there. We made
the \texttt{Nat} class a subclass of the \texttt{Summable<Nat>}.
For class \texttt{Succ} implementation of \texttt{plus} is the same
as before. Spot another slight difference in the return type
of a \texttt{succ} method in class \texttt{Nat} -- it is
declared to be \texttt{Succ}; we will need that to demonstrate
an erasure of covariant method overriding in the result type in
one of the following examples.

Let's use a function \texttt{erasure.Erasure.eraseClass} to generate
the erasure for these classes.

\begin{minted}[mathescape]{java}
class Summable extends Object {
  Summable() { super(); }
  Object plus(Object other_) { return other_; }
}

class Nat extends Summable {
  Nat() { super(); }
  Succ succ() { return new Succ(this); }
}

class Zero extends Nat { 
  Zero() { super(); }
}

class Succ extends Nat {
  Nat prev;
  Succ(Nat prev) { super(); this.prev = prev; }
  Object plus(Object n_) {
    return (Nat)(this.prev.plus((Nat)(n_).succ())); 
  }
}
\end{minted}
What has the erasure changed here?

\begin{itemize}
\item in the class \texttt{Summable} type parameters list was removed
   and all the type variables were replaced with an~\texttt{Object} --
   declared type bound for the \texttt{X} variable (see
   \texttt{X extends Object} in the original class)
\item the class \texttt{Nat} now extends an erased \texttt{Summable}
  class
\item according to the \texttt{plus} method's signature change in
  \texttt{Summable}, the signature of \texttt{plus} in the \texttt{Succ}
  class was adjusted to be identical (modulo argument names);
  two casts to the \texttt{Nat} type were inserted to recover
  the original information about types: first over the access to 
  an \texttt{n\_} variable,
  second over the invocation of~a~method \texttt{plus} that
  happened to return a natural number in the generic version.
\end{itemize}
Now, let's construct a simple expression using these classes. This
will correspond to arithmetic operation $2 + 1$.

\begin{minted}[mathescape]{java}
new Succ(new Succ(new Zero())).plus(new Succ(new Zero()))
\end{minted}
After the erasure it looks almost the same.
\begin{minted}[mathescape]{java}
(Nat)(new Succ(new Succ(new Zero())).plus(new Succ(new Zero())))
\end{minted}

The \texttt{plus} method returns an \texttt{Object}, but
type erasure was smart enough to insert upcast around the invocation
of this method, to recover correct type from the original program.

\paragraph{Example 2} summable lists

Let's review another example -- lists that can contain some
summable elements and compute total \texttt{sum} of
all their elements.

\begin{minted}[mathescape]{java}
class List<X extends Summable<X>> extends Object {
  List() { super(); }
  X sum(X zero) { return zero; }
}

class Nil<X extends Summable<X>> extends List<X> {
  Nil() { super(); }
}

class Cons<X extends Summable<X>> extends List<X> {
  X head;
  List<X> tail;
  Cons(X head, List<X> tail) {
    super(); this.head = head; this.tail = tail;
  }
  X sum(X zero) {
    return this.tail.sum(zero).plus(this.head);
  }
}
\end{minted}
We have base \texttt{List<X>} class and its two subclasses:
\begin{itemize}
\item \texttt{Nil} -- corresponding to an empty list
\item \texttt{Cons} -- list constructor which holds single
  element \texttt{head} of type \texttt{X} and rest of the list
   -- \texttt{tail} of type \texttt{List<X>}
\end{itemize}
For instance, list \texttt{[1, 0]} can be encoded as
the following expression:

\begin{minted}[mathescape]{java}
new Cons<Nat>(new Succ(new Zero()), new Cons<Nat>(new Zero(), new Nil<Nat>()))
\end{minted}

A method \texttt{sum} takes the parameter \texttt{zero} which is
summed into all elements of our list. Overridden occurrence
uses recursive call first to compute sum of \texttt{tail} (it will
return \texttt{X}) and then invokes method \texttt{plus} adding
\texttt{head} element to the final result. Notice that in this
class definition there is no single occurrence of classes \texttt{Nat},
\texttt{Zero} or \texttt{Succ} -- we were able to express
\texttt{sum} operation on list using only the abstract \texttt{plus}
method that we defined for summables.

Let's review erasure of such a summable list implementation.

\begin{minted}[mathescape]{java}
class List extends Object {
  List() { super(); }
  Summable sum(Summable zero_) { return zero_; }
}

class Nil extends List {
  Nil() { super(); }
}

class Cons extends List {
  Summable head;
  List tail;
  Cons(Summable head, List tail) {
    super(); this.head = head; this.tail = tail;
  }
  Summable sum(Summable zero_) {
    return (Summable)(this.tail.sum(zero_).plus(this.head));
  }
}
\end{minted}

Again, all the type parameters were removed and all the occurrences
of type variables were replaced with their bounds -- the 
\texttt{Summable} type. The erasure procedure is
optimized in that way that it doesn't insert the casts, if they are
not absolutely necessary -- see implementations of \texttt{sum} method
and references to \texttt{zero} argument which are not cast.
The only cast we need to insert is placed around the invocation of 
the \texttt{plus} method from \texttt{Summable}, which still returns
an \texttt{Object}.

Having the context of the \texttt{Nat} and \texttt{List} classes,
let's consider such an expression:

\begin{minted}[mathescape]{java}
new Cons<Nat>(
  new Succ(new Succ(new Succ(new Zero()))),
  new Cons<Nat>(
    new Succ(new Succ(new Zero())),
    new Nil<Nat>()
  )
).sum(new Zero())
\end{minted}
...and its erased version:
\begin{minted}[mathescape]{java}
(Nat) new Cons(
  new Succ(new Succ(new Succ(new Zero()))),
  new Cons(
    new Succ(new Succ(new Zero())),
    new Nil()
  )
).sum(new Zero())
\end{minted}

We constructed list of 2 natural numbers
(3 and 2) by instantiating \texttt{Cons}es with the type argument
\texttt{Nat}, which were removed during erasure. Method
\texttt{sum} returns a \texttt{Summable}, but cast to the
\texttt{Nat} class was inserted to ensure that both expressions have
the same types in corresponding type checkers (they both types
to \texttt{Nat}).
Let's evaluate erased expression using the \emph{FJ} evaluator:

\begin{minted}[mathescape]{java}
new Succ(new Succ(new Succ(new Succ(new Succ(new Zero())))))
\end{minted}

As a result, we got encoding of number the 5 which is sum of list
elements (3 and 2) with explicit 0 passed to the \texttt{sum}
invocation.

\paragraph{Example 3} functions as objects

So far we have seen rather simple examples. Now let's try to
encode something more advanced.
We want to encode an interface for unary functions which takes
a single argument of type \texttt{X} and returns a value
of type \texttt{Y}.

\begin{minted}[mathescape]{java}
class UnaryFunc<X extends Object, Y extends Object> extends Object {
  Y ignored;
  UnaryFunc(Y ignored) { super(); this.ignored = ignored; }
  Y apply(X arg) {
    return this.ignored;
  }
}
\end{minted}

We want to represent simple functions as instances of the
\texttt{UnaryFunc} class with single method \texttt{apply} for
computing function value for the given argument. Again, due to lack
of interfaces, we have to provide trivial implementation for
\texttt{apply}. Since we don't require \texttt{Y} as an argument
for a method, the trick is to create a member of the same type as
the function's result type and return it in our trivial implementation.

Let's encode a simple function for natural numbers, $f(n) = 2 * n + 1$.

\begin{minted}[mathescape]{java}
class TwicePlus1 extends UnaryFunc<Nat, Nat> {
  TwicePlus1(Nat ignored) { super(ignored); }
  Succ apply(Nat n) {
    return n.plus(n).succ();
  }
}
\end{minted}

The class \texttt{TwicePlus1} represents that function by replacing
multiplication by 2 with addition of arguments and incrementation
by calling a \texttt{succ}. Notice that since for every natural argument,
result of such a function will be positive number -- we can encode that
within the type system by declaring result as a \texttt{Succ} type,
while still passing \texttt{Nat} as a second type argument to
the \texttt{UnaryFunc}. This is demonstration of aforementioned
\emph{covariant method overriding} in the \emph{FGJ} -- we can declare
the result type of overridden method as a subtype of the result type
of method declared in super class, even if this type was a
type variable. It's the subtyping that takes care of resolving
the type variables and the actual type arguments passed.
That is the reason why we needed the contexts $\Delta$.

Let's review erasure of classes \texttt{UnaryFunc} and
\texttt{TwicePlus1}.

\begin{minted}[mathescape]{java}
class UnaryFunc extends Object {
  Object ignored;
  UnaryFunc(Object ignored) { super(); this.ignored = ignored; }
  Object apply(Object arg_) {
    return this.ignored;
  }
}

class TwicePlus1 extends UnaryFunc {
  TwicePlus1(Object ignored) { super(ignored); }
  Object apply(Object n_) {
    return (Nat)((Nat)(n_).plus((Nat)n_)).succ();
  }
}
\end{minted}

The same as before, generic types were removed from the classes
and replaced with their bounds -- \texttt{Object}s. Covariant
method overriding is not present in the \emph{FJ}, so erasure had
to ensure that types in methods signatures in both classes are
identical. Proper casts were inserted in overridden method
\texttt{apply}:

\begin{itemize}
\item two casts around the reference to a variable \texttt{n\_} --
  to recapture its type, being \texttt{Nat} in the example with
  generic types
\item cast to \texttt{Nat} around invocation of method
  \texttt{plus}, as well as in the previous examples.
\end{itemize}
\textbf{Combining it together}

Let's extend the \texttt{List} class to support mapping its
elements with unary functions.

\begin{minted}[mathescape]{java}
class List<X extends Summable<X>> extends Object {
  ...
  <Y extends Summable<Y>> List<Y> map(UnaryFunc<X, Y> f) {
    return new Nil<Y>();
  }
}

class Cons<X extends Summable<X>> extends List<X> {
  ...
  <Y extends Summable<Y>> List<Y> map(UnaryFunc<X, Y> f) {
    return new Cons<Y>(f.apply(this.head), this.tail.map(f));
  }
}
\end{minted}

In the base class we added method \texttt{map}, parameterized with
the type parameter \texttt{Y} that takes unary function and simply
constructs an empty list of summables \texttt{Y}. In \texttt{Cons}
we return a new list with function \texttt{f} applied to the
\texttt{head} element and \texttt{tail} mapped by \texttt{f}.

How the erasure of the added methods looks like?

\begin{minted}[mathescape]{java}
class List extends Object {
  ...
  List map(UnaryFunc f_) { return new Nil(); }
}

class Cons extends List {
  ...
  List map(UnaryFunc f_) {
    return new Cons(
      (Summable)(f_.apply(this.head)),
      this.tail.map(f_)
    );     
  }
}
\end{minted}

\texttt{UnaryFunc} in the \texttt{map}'s argument occurs in erased
version. Then cast to \texttt{Summable} was inserted around
\texttt{apply} invocation in the \texttt{Cons} class.

Finally, let's construct an example program which uses all the classes
defined so far.

\begin{minted}[mathescape]{java}
new Cons<Nat>(
  new Succ(new Zero()),
  new Cons<Nat>(
    new Succ(new Succ(new Zero())),
    new Nil<Nat>()
  )
).map<Nat>(new TwicePlus1(new Zero()))
 .sum(new Zero())
\end{minted}

We construct list \texttt{[1,2]}, map it by function
\texttt{TwicePlus1} and sum all elements of resulting list
with~0. Erased version contains only topmost cast to \texttt{Nat}
(remember, \texttt{sum} result type was \texttt{Summable},
but we have concrete subclass here).

\begin{minted}[mathescape]{java}
(Nat)(
  new Cons(
    new Succ(new Zero()),
    new Cons(
      new Succ(new Succ(new Zero())),
      new Nil()
    )
  ).map(new TwicePlus1(new Zero()))
   .sum(new Zero())
)
\end{minted}

Erased program evaluates to the encoding of a number \texttt{8},
as we expected.

\begin{minted}[mathescape]{java}
new Succ(new Succ(new Succ(new Succ(new Succ(new Succ(new Succ(new Zero())))))))
\end{minted}


\section{Erasure properties}

We have seen the type erasure in action on programming language,
which although simplified to bare minimum, is able to encode,
type-check and evaluate quite advanced examples. We reviewed
erasure of all examples and saw types of some of them and
they corresponded to types found by generic type-checker. Moreover,
erased programs behaved exactly as we expected when we were
defining their generic version. Is it a matter of convenient
examples, or is it a kind of general property?
Authors of \cite{fj} come with an answer, stating several theorems.

\newtheorem{theorem}{Property}
\begin{theorem}[Erasure preserves typing]
For all well-typed \emph{FGJ} class tables, they are well-typed
after erasing under \emph{FJ} typing rules.
\end{theorem}
This property ensures us that the \emph{FGJ} is fully compatible
superset of \emph{FJ} regarding type-checking.

\begin{theorem}[Erasure preserves execution results]
If a well-typed \emph{FGJ} program evaluates to some value $w$
in type-passing semantics, then erased program evaluates to
the erasure of value $w$ in the \emph{FJ} evaluator.
\end{theorem}

Both theorems are proved in \cite{fj}. There are some technical
difficulties in proving the second theorem, connected with an
insertion of special synthetic casts during erasure. Finally,
behaviour of the program after erasure is equivalent modulo
evaluation of synthetic casts.

\section{Conclusion}

We have discussed one of possible implementation of generic
types -- \emph{type erasure}. There are several known problems
in programming languages and development platforms built on top
of the idea of erasing generic types, amongst which the most
popular is \emph{Java Virtual Machine}. Deep understanding of
the pure idea and ability to review although simplified,
yet working implementation will allow the reader to better
understand consequences of the limitations and their real roots. 

\begin{thebibliography}{1}

  \bibitem{fj} A. Igarashi, B. C. Pierce, P. Wadler
  {\em Featherweight Java: A Minimal Core Calculus for Java and GJ}.
  ACM Transactions on Programming Languages and Systems (TOPLAS), 23(3), May 2001.

  \bibitem{pjzo} P. Krzemiński Slides from {\em Fundamentals of
  Object Oriented Languages} seminary lead by D. Biernacki.
  Department of Computer Science, University of Wrocław, 2014.
  Available at
  \url{http://www.ii.uni.wroc.pl/~dabi/courses/PJZO14/pkrzeminski/fj.pdf}.

\end{thebibliography}


\end{document}
